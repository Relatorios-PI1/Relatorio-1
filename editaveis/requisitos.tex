% \part{Análise de Requisitos}
\chapter[Análise de Requisitos]{Análise de Requisitos}

Devido à alta demanda de energia elétrica no campus FGA e também ao constante desperdício de energia elétrica promovida por meio de ações ou por equipamentos mal dimensionados, tem-se o problema da constante quebra no contrato de demanda de energia elétrica feito entre a companhia de energia elétrica e a UnB. Esse problema tem gerado gastos e dívidas entre as duas instituições, que poderiam ser utilizados para fins de projetos ou até na infraestrutura do campus. Além do desperdício e a alta demanda de energia elétrica, a condição ambiental existente no campus, bem como a arquitetura dos prédios construídos, são aspectos que devem ser levados em conta na questão de sustentabilidade energética, por não aproveitar a iluminação diária ou por não propiciar a circulação adequada de ar, provocando um aumento no uso de lâmpadas e de climatizadores em diversos pontos do campus.

A partir dos principais problemas no âmbito de consumo e gerenciamento de energia elétrica no campus, identificados com diagramas de Fishbone e, com base na pesquisa das condições insustentáveis quanto ao uso de energia elétrica no campus FGA levantadas por meio de entrevistas e conversas sobre o problema com professores e usuários do campus, propõe-se os seguintes requisitos relativos ao projeto.

\begin{description}
	\item [Requisitos funcionais] \

	\begin{itemize}
	\item Instalação de fontes alternativas e complementares de energia elétrica.
	\item Dispositivos de gerenciamento sobre gastos energéticos em tempo real.
	\item Instalações de sistemas “on grid” integrando as instalações com a distribuidora de energia.
	\item Automação predial eletro-eletrônica nos principais prédios do campus.
	\end{itemize}

	\item[Requisitos não funcionais] \
	
	\begin{itemize}
	\item Controle de iluminação do prédio deverá ser monitorada por ferramentas de dimerização automática e deverá conter sensores de presença em locais de grande fluxo de pessoas.
	\item Utilização de inversores “grid tie” interligados às fontes alternativas e à rede elétrica, geridos por medidores bidirecionais.
	\item Medidores digitais de consumo inteligente, com informações em tempo real sobre o consumo, potência e qualidade dos serviços fornecidos.
	\end{itemize}
\end{description}

\section{Análise para fontes renováveis}
Para Fontes renováveis de energia o sistema deve apresentar-se como complemento à rede elétrica disponível da concessionária de energia elétrica, formando uma rede de smart grid não auto-sustentável, caracterizada pela diminuição do gasto energético e podendo assim facilitar a renegociação de dívidas com as concessionárias.

Dentre as fontes alternativas de energia avaliadas, destaca-se a energia solar (a partir de painéis fotovoltaicos), a energia heliotérmica (proveniente da concentração de feixes solares), a energia eólica e a energia de biomassa (a partir de um biodigestor). 

\begin{description}
	\item [Fatores na instalação] \

	\begin{itemize}
	\item O espaço físico é um problema inicial deste requisito, assim como a fonte será utilizada para gerar energia.
	\end{itemize}
	\item[Problemas associados à instalação]\
	\begin{itemize}
	\item Biodigestor: O cheiro que é exalado por este meio acabaria incomodando a todos que estivessem perto desta fonte, outro fator é que seria interessante a fonte estar relativamente próxima ao Restaurante Universitário (RU) na universidade pois a maior produção de objetos orgânicos é neste local.
	\item Painéis fotovoltaicos: Cálculo do dimensionamento do gerador fotovoltaico para uma estrutura como a FGA, materiais que possam transmitir a energia sem perda e que sejam baratos para a instalação, o formato da superfície ao qual os painéis serão instalados, pesquisa prévia à instalação para ver o índice de radiação e quais os melhores pontos a serem abordados para a construção.
	\end{itemize}
\end{description}

\section{Análise para Medidores digitais de consumo inteligente, com informações em tempo real sobre o consumo, potência e qualidade dos serviços fornecidos}
O Dispositivo de gerenciamento da energia elétrica (DGEE) é basicamente um quadro elétrico trifásico, de um controlador lógico programável (CLP) e um supervisório para monitorar os valores eficazes da corrente elétrica e da tensão elétrica, o fator de potência, demanda e consumo da energia elétrica, usando objeto de ligação embarcado para controle de processo, que permite ao usuário acessar em tempo real os parâmetros da CLP. A comunicação entre o DGEE e os servidores serão feitas por uma rede sem fio.

Por conta dos elevados custos da geração e da distribuição de energia elétrica para o Campus da FGA, e escassez de recursos naturais secundários, o uso da energia elétrica é de grande importância. No entanto a possibilidade de gerenciar e controlar, pode trazer inúmeros benefícios. Ter a possibilidade do monitoramento e armazenamento dos dados, tais como fator potência, ativa, reativa, e valores efetivos dos picos de tensões e correntes a fim de reduzir o desperdício e o custo de consumo. 
 
Com a demanda exigida pela concessionária contratada, torna se necessário o uso do dispositivo de gerenciamento de energia elétrica.
O circuito para implementação é dividido em duas partes: Circuito de potência e circuito de controle.
O circuito de potência é responsável pela interface com a rede elétrica para a aquisição dos dados de energia elétrica (corrente, tensão, potência ativa e reativa e fator de potência). Já o circuito de controle é o responsável pelo tratamento dos dados medidos e pela comunicação via Wireless com o servidor.

\section{Instalações de sistemas “on grid” integrando as instalações com a distribuidora de energia}
Ao contrário de sistemas auto-suficientes, que são sistemas compostos por baterias e conexões exclusivas a uma fonte alternativa, os sistemas de conexão à rede (ou sistemas “on grid”) podem ser utilizados tanto para abastecer uma residência, quanto para simplesmente produzir e injetar a energia na rede elétrica, assim como uma usina hidroelétrica ou térmica, porém de forma complementar.

Se o proprietário do sistema produzir mais energia do que consome, a energia produzida fará com que o medidor “gire para trás”. Quando produzir menos do que consome, o medidor deverá “girar mais devagar”. Vale observar que o medidor deve ser apropriado para contabilizar o fluxo de energia nos dois sentidos.

\section{Utilização de inversores “grid tie” interligados às fontes alternativas e à rede elétrica, geridos por medidores bidirecionais}
Inversor para conexão à rede (ou grid connected inverter) é um dispositivo eletrônico que permite aos usuários de energia fotovoltaica ou eólica interligar seus sistemas com a rede da concessionária e injetar na rede o excedente de energia produzido pelos sistemas (fotovoltaico ou eólico). 

A principal característica de um inversor para conexão à rede é a capacidade de se interligar com a rede da concessionária, sincronizando sua frequência (60 Hz, no Brasil) e tensão de saída (CA) com a mesma, e se desconectar da rede quando esta deixa de fornecer energia como, por exemplo, devido a desligamento para reparo ou falha na rede. 

Para gerar a onda senoidal pura que seja compatível com a onda de corrente alternada produzida por sua companhia de distribuição, são necessários inversores de onda senoidal. Estes são um pouco mais caros, mas eles são compatíveis com quase todos os equipamentos e aparelhos que pode ser operado com a alimentação da rede. Estes inversores, classificados como multifuncionais, também permitem que você obtenha informações a respeito do excesso de energia. Inversores Grid-tie pode ser usado para manter um banco de bateria back-up cobrado por energia de emergência em caso de falta de energia elétrica.
