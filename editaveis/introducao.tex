\chapter[Introdução]{Introdução}
% \addcontentsline{toc}{chapter}{Introdução}

Persuasão. Esse é um dos pontos essenciais para a realização de um bom projeto. A capacidade em gerar argumentação e reinventar algo ao usar métodos estrategicamente pensados. Tanto é que é um dos tópicos (e que se diria, um dos mais importantes) de um ponto de controle.

Esse relatório – estudo, já possui em si o nome de seu principal propósito, “controle”, e possui uma sequência de finalidades. No entanto, o que o poderia abreviar seu intuito em poucas palavras seria a organização, comunicação, análise e gerenciamento.

Em cada ponto desses, está a chave para o sucesso. A comunicação é um ponto essencial, que abraça todas as ideias e respectivas resoluções de problemas.  O gerenciamento, só é possível graças a comunicação, e ambos, só seguem em frente graças a análise engenhosa e minuciosa de cada ponto que envolve um projeto.  

Com esses pontos estratégicos, e suas devidas subdivisões, a proposta está devidamente encaminhada.  Se direciona a um futuro promissor que agrega tanto o bem profissional aos envolvidos no projeto, quanto o bem daqueles que foram beneficiados pelo projeto em si.