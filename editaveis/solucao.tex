% \part{Solução Geral}
\chapter[Solução Geral]{Solução Geral}

A proposta do projeto de um modelo de Smart Grid para a FGA surgiu a partir da necessidade de reduzir os gastos energéticos da FGA, diminuir as perdas energéticas, aumentar a sustentabilidade para o meio ambiente que o cerca e diminuir a dependência da concessionária de energia elétrica. Além dos aspectos anteriormente citados, o projeto visa melhorar a qualidade do sinal elétrico que circula na FGA, tornando a rede mais estável e monitorando a quantia de energia gasta em vários pontos para evitar desperdícios e permitir uma administração inteligente no uso de energia elétrica.

Foram escolhidas pela equipe duas fontes alternativas de energia para trabalhar paralelamente na produção de energia para o Campus e evitar a total dependência por parte de fornecimento energético externo, são elas a fotovoltaica e o biogás. Estas fontes foram escolhidas através de pesquisas e estudos de viabilidade para implantação e dimensionamento no Campus. Vale ressaltar que a demanda energética será maior que a energia produzida pelas fontes renováveis, e ainda será necessário a utilização de energia elétrica proveniente da CEB.

\section{Fotovoltaica}
Energia fotovoltaica é a energia elétrica produzida a partir de luz solar, e pode ser produzida mesmo em dias nublados ou chuvosos. Quanto maior for a radiação solar maior será a quantidade de eletricidade produzida. O processo de conversão da energia solar utiliza células fotovoltaicas (normalmente feitas de silício ou outro material semicondutor). Quando a luz solar incide sobre uma célula fotovoltaica, os elétrons do material semicondutor são postos em movimento, desta forma gerando eletricidade. A energia fotovoltaica é uma tecnologia 100\% comprovada, sistemas fotovoltaicos conectados a á rede elétrica já são utilizados a mais de 30 anos.

\begin{description}
	\item [Vantagens] \

	\begin{itemize}
		\item A energia solar não polui durante seu uso. A poluição decorrente da fabricação dos equipamentos necessários para a construção dos painéis solares é totalmente controlável utilizando as formas de controlo existentes actualmente.
		\item As centrais necessitam de manutenção mínima.
		\item Os painéis solares são a cada dia mais potentes ao mesmo tempo que seu custo vem decaindo. Isso torna cada vez mais a energia solar uma solução economicamente viável.
		\item A energia solar é excelente em lugares remotos ou de difícil acesso, pois sua instalação em pequena escala não obriga a enormes investimentos em linhas de transmissão.
		\item Em países tropicais, como o Brasil, a utilização da energia solar é viável em praticamente todo o território, e, em locais longe dos centros de produção energética sua utilização ajuda a diminuir a procura energética nestes e consequentemente a perda de energia que ocorreria na transmissão.
	\end{itemize}

	\item [Desvantagens] \

	\begin{itemize}
		\item Existe variação nas quantidades produzidas de acordo com a situação climática (chuvas, neve), além de que durante a noite não existe produção alguma, o que obriga a que existam meios de armazenamento da energia produzida durante o dia em locais onde os painéis solares não estejam ligados à rede de transmissão de energia.
		\item As formas de armazenamento da energia solar são pouco eficientes quando comparadas por exemplo aos combustíveis fósseis (carvão, petróleo e gás), e a energia hidroeléctrica (água).
	\end{itemize}
\end{description}

\section{Biogás}
A compreensão do biogás como fonte renovável de energia traz como necessário o entendimento da biomassa como recurso com potencial energético. Biomassa poder ser designada como a massa total de matéria orgânica acumulada num espaço. Desta forma pertencem à biomassa todos os vegetais e animais, bem como os seus resíduos. Além disso, os resíduos industriais dos segmentos madeireiro e alimentício e os resíduos urbanos, como esgoto doméstico. Esta abrangência de biomassa pode ser transformada pelas tecnologias convenientes de conversão em biocombustíveis e energias térmica, mecânica e elétrica (STAISS \& PEREIRA, 2001).

O biogás é produzido a partir da digestão anaeróbia de matéria orgânica. Basicamente o processo é constituído pela aglomeração destes resíduos em uma estrutura fechada, denominada biodigestor. No biodigestor as bactérias inerentes aos dejetos obtêm, dentro de condições adequadas de trabalho, suas energias a partir da atuação fermentativa nos resíduos orgânicos, que traz como produtos o biogás, o efluente líquido mineralizado (após tratamento) e biofertilizantes.

Os microrganismos atuantes neste processo precisam de condições adequadas para a eficiência do trabalho fermentativo, como baixo teor de substâncias tóxicas e poder calorífico adequado da matéria orgânica, temperatura na faixa de 30-35 o C e pH entre 7-7,5. Essas restrições, junto a escolha adequada do tipo de biodigestor, não expõem os microrganismos a condições estressantes, promovendo um bom aproveitamento da aglomeração orgânica.

O biodigestor é o local onde a matéria orgânica é depositada e sofre a digestão anaeróbia bacteriana. Esta construção é basicamente constituída por um canal de entrada de resíduos, uma câmara de digestão, um canal de remoção do biofertilizantes, por um desnível coletor dos efluentes líquidos e por uma canalização para saída do gás. O biodigestor do tipo indiano é designado por este projeto, em virtude da sua simplicidade tecnológica e do posicionamento subterrâneo da sua câmara de digestão, que contribui na regulação das condições térmicas da atuação bacteriana.

O produto de maior relevância, neste estudo, a ser obtido é o biogás. Este por sua vez, é uma mistura de outros gases cujas características qualitativas e quantitativas dependem dos tipos residuais postos à atividade de digestão anaeróbia. Normalmente o gás metano (CH 4 ) aparece como constituinte em maior percentual no biogás – de 50 a 80\% (LA FARGE, 1979). A relevância do biogás, aqui, é devido ao interesse no produto final energia elétrica. A energia química do gás, por um processo controlado de combustão, é convertida em energia mecânica que ativa um gerador elétrico.

A biomassa adotada neste estudo é a proveniente de resíduos orgânicos produzidos pelo restaurante universitário do Campus Faculdade do Gama, da Universidade de Brasília, e o lixo orgânico produzido pela cidade do Gama, Distrito Federal. O objetivo é a redução da dependência energética da concessionária de energia elétrica da região, o incentivo ao estudo e desenvolvimento de fontes renováveis de energia que podem ser estendidas a regiões deficientes de infraestrutura elétrica, e a realização de um melhor de manejo de resíduos domésticos, que permite um controle mais adequado de disposição final de resíduos e aliviar a sobrecarga dos lixões do Distrito Federal.
